\section{Conclusion \& avenir}
\label{sec:ccl}
Ce stage aura été formateur de bien des manières et je vais tenter de mettre en avant les principales ici. 

Découvrir l'intégration continue et ses applications à la sécurité, tout d'abord : c'est une notion dont j'avais entendu parler, et qui revenait régulièrement dans les offres de stages que j'ai parcourues, mais qui restait flou à mes yeux et qui s'est avérée, finalement, très intéressante à mettre en pratique. 

L'intégration continue est une branche encore jeune de l'informatique en entreprise, et, semble-t-il, porteuse d'avenir, mais c'est aussi un domaine intéressant du point de vue de la sécurité tant il offre d'opportunités, que ce soit comme ce que j'ai pu faire automatiser des analyses de sécurité ou d'une manière générale auditer et améliorer le processus de CI en lui-même d'un point de vue cybersécurité (après tout, il s'agit d'accéder au net, éventuellement de télécharger ou téléverser des documents voire déployer une application, ou se connecter à distance à des machines... tout cela peut présenter des failles de sécurité et requérir des experts en la matière pour les éviter/corriger). 

Me déplacer en personne chez un client, mener un audit, ensuite, a été nouvelle. J'ai pu participer à ce projet de son début (la planification de l'intervention, des différentes réunions avec le client, la répartition des tâches dans notre équipe) à sa fin (la rédaction des livrables et de la documentation) tout en intervenant évidemment sur la partie technique, les analyses à proprement parler. 

Se plonger momentanément dans la façon de travailler, les méthodologies d'une entreprise dont on n'a pas l'habitude et dont les méthodes de travail et les besoins sont radicalement différents de ceux d'Alter Frame auxquels je m'étais habitué était d'une certaine manière déroutant, et d'une autre intéressant. 

En fin de compte le seul bémol que je vois est de ne pas avoir eu le temps de mener aussi loin que je l'aurais aimé la partie CI du stage, mais celle-ci ne sera pas perdue dans la mesure où le code reste disponible pour des interventions futures, correctement commenté et documenté. 

Pour rester sur la partie CI, et mettre un point final à ce rapport, on m'a proposé de rejoindre Alter Defense \& Security (voir la figure~\ref{fig:filiales} pour rappel) après la fin de mon stage, comme consultant en sécurité informatique, et il se trouve que mon tout premier contrat va être en lien avec l'intégration continue puisqu'il va s'agir de mettre en place et sécuriser (en utilisant des scripts en Python) l'intégration et le déploiement continus d'une solution de threat intelligence\cite{threat_intell_wiki} sur le Cloud pour un client d'Alter Defense.

\pagebreak
\section{Crédits}
\begin{itemize}
\item Le template utilisé pour la mise en forme de ce document est l'oeuvre d'Andrew Hobbs, disponible \href{https://www.overleaf.com/latex/templates/climate-policy-initiative-report-template/kjfjzrcjgtqg#.WTVoYKJVtv0}{ici} sous licence \href{https://creativecommons.org/licenses/by/4.0/}{Creative Commons Attribution 4.0 International}.  
\item Le graphique~\ref{fig:ci_process} utilise une icône faite par \href{https://www.flaticon.com/authors/situ-herrera}{Situ Herrera} disponible sur \href{https://www.flaticon.com}{www.flaticon.com} sous licence \href{https://creativecommons.org/licenses/by/3.0/}{Attribution 3.0 Unported}.
\end{itemize}