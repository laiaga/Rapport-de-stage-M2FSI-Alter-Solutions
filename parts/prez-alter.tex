\section{Présentation d'Alter Solutions Engineering} 
Alter Solutions Engineering est l'entreprise qui m'a accueilli pour la durée de mon stage de fin d'études, nous allons donc commencer par la présenter rapidement.

\subsection{Historique d'Alter Solutions Engineering}
Alter Solutions Engineering est une entreprise relativement jeune : elle a été créée en 2006 et, si elle n'entre plus maintenant dans la catégorie des PME en termes de nombre de collaborateurs, elle reste une structure de petite taille.

Le siège social de l'entreprise se trouve à Versailles et c'est là où travaille l'équipe de développement française dont je faisais partie. En pratique, il s'agissait de l'équipe de développement d'Alter Frame qui est une entité enfant d'Alter Solutions Engineering (voir section~\ref{subsec:subdivisions}).

Bien qu'Alter Frame ait des clients et des domaines d'intervention variés, en termes de technologes il y a trois pôles de compétences qui sont caractéristiques de l'entreprise et reviennent le plus régulièrement :
\begin{itemize}[label=$\bullet$]
\item Java ;
\item .NET ;
\item PHP.
\end{itemize}

Mon stage ne s'est pas cantonné au domaine du développement mais il en a tout de
même inclus celui-ci s'est déroulé au sein de l'équipe Java.

\subsection{Les subdivisions d'Alter Solutions et leurs secteurs d'activité}
\label{subsec:subdivisions}
Alter Solutions est une société de conseil en hautes technologies mais en pratique, elle est composée de trois filières qui ont chacune une spécialité bien distinctes.

\subsubsection{Alter Solutions}
Cette filiale est spécialisée dans le conseil en ingénierie, notamment dans les domaines de l'énergie, des transports, des sciences de la vie, du digital et du numérique.

  \subsubsection{Alter defence \& security}
  Alter defence est également orientée vers le conseil, mais cette fois plus particulièrement dans l'aéronautique, le sptial et la défense.
  
  \subsubsection{Alter Frame}
  Alter Frame enfin est la branche spécialisée dans l'édition de logiciels et celle que j'ai rejoint durat mon stage. C'est une ESN\footnote{Entreprise de Services du Numérique, cf. \url{https://fr.wikipedia.org/wiki/Entreprise_de_services_du_num\%C3\%A9rique}} dont l'activité est elle-même répartie en deux catégories :
\begin{itemize}[label=$\bullet$]
    \item le conseil, c'est-à-dire le fait de fournir des spécialistes d'un domaine du numérique pour la durée d'un contrat à un client ;
    \item le développement de logiciels au forfait, c'est-à-dire le fait de prendre commande d'un logiciel à réaliser en interne et de le livrer à la fin du contrat.
    \end{itemize}

\subsection{Quelques (derniers) chiffres}

