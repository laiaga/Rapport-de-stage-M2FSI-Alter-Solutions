\section{Introduction}
Développement Java et développement d'une solution d'analyse statique de sécurité : ce sont les deux branches de mon stage. Il s'agit pour partie de prendre part aux contrats en Java d'Alter Frame, l'entreprise qui m'accueille pour la durée du stage, et d'autre part d'intervenir sur un projet en interne visant à mettre en place une analyse de sécurité systématique des projets Web au-travers de pratiques d'intégration continue\footnote{Continuous Integration ou intégration continue, cf. \url{https://fr.wikipedia.org/wiki/Int\%C3\%A9gration_continue}}. La présentation d'Alter Frame sera donc naturellement la toute première partie de ce rapport. 

Ce sujet a l'avantage d'être ouvert et diversifié. Il me permet d'une part de travailler sur du pur développement et d'autre part de mettre en pratique la composante sécurité de la formation FSI\footnote{Fiabilité et sécurité informatique, cf. \url{http://masterinfo.univ-mrs.fr/FSI.html}}, tout en découvrant les concepts de CI qui m'étaient jusque là étrangers, ainsi que des technologies qui vont de pair telles que Docker. Présenter ces deux pans de mon travail à Alter Frame composera la suite du rapport.

Celui-ci se cloturera en analysant et résumant les apprentissages que j'ai retiré de ce stage, et les perspectives d'avenir qu'il m'ouvre, aussi bien chez Alter Frame si l'entreprise m'offre de continuer mon travail après le stage qu'ailleurs dans l'éventualité contraire. 