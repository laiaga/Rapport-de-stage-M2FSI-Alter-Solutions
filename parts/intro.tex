\section*{Introduction}
Développement Java et développement d'une solution d'analyse statique de sécurité : ce sont les deux principales branches de mon stage. Il s'agit pour partie de prendre part aux contrats en Java d'Alter Frame, l'entreprise qui m'accueille pour la durée du stage, et d'autre part d'intervenir sur un projet en interne visant à mettre en place une analyse de sécurité systématique des projets Web au-travers de pratiques de CI\cite{ci_wiki}\footnote{Continuous Integration ou intégration continue}. 

Ce sujet a l'avantage d'être ouvert et diversifié. Il me permet d'une part de travailler sur du pur développement et d'autre part de mettre en pratique la composante sécurité de la formation FSI\cite{fsi}\footnote{Fiabilité et sécurité informatique}, tout en découvrant les concepts de CI qui m'étaient jusque là étrangers, ainsi que des technologies qui vont de pair telles que Docker.

En pratique, un troisième pan viendra s'ajouter à mon sujet de stage : un audit technique pour un client d'Alter Frame souhaitant des pistes d'amélioration de son application, notamment en termes de qualité de code. 

À noter que, par discrétion à leur égard, les noms des clients d'Alter Frame ne seront pas mentionnés et seront effacés des captures d'écran que vous trouverez dans ce document. Il en ira de même pour les différents projets.